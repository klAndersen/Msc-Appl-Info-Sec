\section{Attack against a substitution cipher}
\label{section:atk_subst_cipher}

\subsection{Amount of letters (A-Z) found in the ciphertext}
\label{section:1a}
\begin{center}
	\begin{tabular}{ | c | c | c | } \hline 
		Character & Absolute amount & Relative amount \\ \hline
		A & 22 & 1.51\%. \\ \hline
		B & 104 & 7.14\%. \\ \hline
		C & 122 & 8.37\%. \\ \hline
		D & 119 & 8.17\%. \\ \hline
		E & 43 & 2.95\%. \\ \hline
		F & 2 & 0.14\%. \\ \hline
		G & 70 & 4.80\%. \\ \hline
		H & 37 & 2.54\%. \\ \hline
		I & 0 & 0.00\%. \\ \hline
		J & 3 & 0.21\%. \\ \hline
		K & 58 & 3.98\%. \\ \hline
		L & 113 & 7.76\%. \\ \hline
		M & 31 & 2.13\%. \\ \hline
		N & 6 & 0.41\%. \\ \hline
		O & 19 & 1.30\%. \\ \hline
		P & 51 & 3.50\%. \\ \hline
		Q & 11 & 0.75\%. \\ \hline
		R & 94 & 6.45\%. \\ \hline
		S & 34 & 2.33\%. \\ \hline
		T & 32 & 2.20\%. \\ \hline
		U & 131 & 8.99\%. \\ \hline
		V & 45 & 3.09\%. \\ \hline
		W & 166 & 11.39\%. \\ \hline
		X & 81 & 5.56\%. \\ \hline
		Y & 61 & 4.19\%. \\ \hline
		Z & 2 & 0.14\%. \\ \hline
	\end{tabular}
\end{center}

\subsection{Decrypting the ciphertext and c) Alphabetic substitution table}
\label{section:1bc}
For both getting the amount, frequency and decrypted ciphertext, I wrote a program in C\footnote{
	The program can be found on my GitHub (substitution\_attack): \\ 
	\url{https://github.com/klAndersen/Msc-Appl-Info-Sec/tree/master/Module\%201\%20-\%20Intro\%20to\%20Crypto/Uebung_01}
}. When mapping the frequency from the ciphertext to the frequency of English letters, the following table emerged. 
\begin{center}
	\begin{tabular}{ | c | c | c | } \hline 
		A$_{plaintext}$ & $\rightarrow$ & U$_{ciphertext}$ \\ \hline		 
		E & $\rightarrow$ & W \\ \hline 
		T & $\rightarrow$ & U \\ \hline 
		A & $\rightarrow$ & C \\ \hline 
		O & $\rightarrow$ & D \\ \hline 
		I & $\rightarrow$ & L \\ \hline 
		N & $\rightarrow$ & B \\ \hline 
		S & $\rightarrow$ & R \\ \hline 
		H & $\rightarrow$ & X \\ \hline 
		R & $\rightarrow$ & G \\ \hline 
		D & $\rightarrow$ & Y \\ \hline 
		L & $\rightarrow$ & K \\ \hline 
		C & $\rightarrow$ & P \\ \hline 
		U & $\rightarrow$ & V \\ \hline 
		M & $\rightarrow$ & E \\ \hline 
		W & $\rightarrow$ & H \\ \hline 
		F & $\rightarrow$ & S \\ \hline 
		G & $\rightarrow$ & T \\ \hline 
		Y & $\rightarrow$ & M \\ \hline 
		P & $\rightarrow$ & A \\ \hline 
		B & $\rightarrow$ & O \\ \hline 
		V & $\rightarrow$ & Q \\ \hline 
		K & $\rightarrow$ & N \\ \hline 
		X & $\rightarrow$ & J \\ \hline 
		J & $\rightarrow$ & F \\ \hline 
		Q & $\rightarrow$ & Z \\ \hline 
		Z & $\rightarrow$ & I \\ \hline 
	\end{tabular}
\end{center}
However, when attempting to decrypt the text, the text was not readable (see Listing \ref{lst:decrypt_c_program}). 
\begin{lstlisting}[caption={Attempt at decrypting using developed C-program}, label={lst:decrypt_c_program}] 
STRRI?
LSW AF O STDT? LSAE'H FW PUDPIHT ON ROGT?

LSAE CI O FTAN BW LSI AF O?

YARF CILN, MTE  A  MDOP  NIL  ...  IS!  ESOH  OH  AN  ONETDTHEONM
HTNHAEOIN,  LSAE  OH  OE?  OE'H  A  HIDE IG ... WALNONM, EONMRONM
HTNHAEOIN ON FW ... FW  ...  LTRR  O  HUPPIHT  O'C  BTEETD  HEADE
GONCONM  NAFTH  GID  ESONMH OG O LANE EI FAKT ANW STACLAW ON LSAE
GID EST HAKT IG LSAE O HSARR YARR AN ADMUFTNE O  HSARR  YARR  EST
LIDRC, HI RTE'H YARR OE FW HEIFAYS.

MIIC. IIIIS, OE'H MTEEONM QUOET HEDINM.  ANC  STW,  LSAE'H  ABIUE
ESOH  LSOHERONM  DIADONM HIUNC MIONM PAHE LSAE O'F HUCCTNRW MIONM
EI YARR FW STAC? PTDSAPH O YAN YARR ESAE ... LONC! OH ESAE A MIIC
NAFT?  OE'RR CI ... PTDSAPH O YAN GONC A BTEETD NAFT GID OE RAETD
LSTN O'VT GIUNC IUE LSAE OE'H GID.  OE  FUHE  BT  HIFTESONM  VTDW
OFPIDEANE  BTYAUHT ESTDT YTDEAONRW HTTFH EI BT A STRR IG A RIE IG
OE. STW! LSAE'H ESOH ESONM? ESOH ... RTE'H YARR OE A EAOR - WTAS,
EAOR.  STW! O YAN YAN DTARRW ESDAHS OE ABIUE PDTEEW MIIC YAN'E O?
LIL! LIL! ESAE GTTRH MDTAE! CITHN'E HTTF EI AYSOTVT VTDW FUYS BUE
O'RR PDIBABRW GONC IUE LSAE OE'H GID RAETD IN. NIL - SAVT O BUORE
UP ANW YISTDTNE POYEUDT IG ESONMH WTE?

NI. NTVTD FONC, STW, ESOH OH DTARRW TXYOEONM, HI  FUYS  EI  GONC  IUE
ABIUE,  HI  FUYS  EI  RIIK  GIDLADC  EI,  O'F  QUOET  COJJW  LOES
ANEOYOPAEOIN ... ID OH OE EST LONC?

ESTDT DTARRW OH A RIE IG ESAE NIL OHN'E OE?

ANC LIL! STW! LSAE'H ESOH ESONM HUCCTNRW YIFONM EILADCH  FT  VTDW
GAHE?  VTDW  VTDW GAHE. HI BOM ANC GRAE ANC DIUNC, OE NTTCH A BOM
LOCT HIUNCONM NAFT ROKT ... IL ... IUNC  ...  DIUNC  ...  MDIUNC!
ESAE'H OE! ESAE'H A MIIC NAFT - MDIUNC!

O LINCTD OG OE LORR BT GDOTNCH LOES FT?

ANC EST DTHE, AGETD A HUCCTN LTE ESUC, LAH HORTNYT.

YUDOIUHRW TNIUMS, EST INRW ESONM ESAE LTNE ESDIUMS  EST  FONC  IG
EST BILR IG PTEUNOAH AH OE GTRR LAH IS NI, NIE AMAON. FANW PTIPRT
SAVT HPTYURAETC ESAE OG LT KNTL TXAYERW LSW EST BILR IG  PTEUNOAH
SAC ESIUMSE ESAE LT LIURC KNIL A RIE FIDT ABIUE EST NAEUDT IG EST
UNOVTDHT ESAN LT CI NIL.

HIUDYT: SEEP://LLL.YRTADLSOETROMSE.IDM/SOEYS/SSMEEM.EXE
\end{lstlisting}
When looking at the ciphertext, one thing that really stood out was the last line. 
This clearly was a link, so I therefore started to map the letters by hand for the link. 
The question now was, what could the text before the link be? 
By using a text editor, and replacing all the letters I had found via the link, I looked for singular occurences. 
Specifically, the apostrophes at the end of the words. 
In most English sentences, this is usually an 's' (e.g. "it's", "what's", etc.). 
\vspace{0.5em}\newline
I also took another look at the listing that I got from the mapping. 
When looking at the listing, the mapping was done based on 'E' being the most used letter. 
However, compared to the manual decryption, 'T' was the value which the letter 'E' was mapped to. 
I therefore went on the assumption that for this text, plaintext letter 'E' was secondary, which meant that 'E' was represented by 'T's cipher letter.
I did the same for the plaintext letter 'A', which gave me the partially readable ciphertext shown in Listing \ref{lst:partial_decryption}.

\begin{lstlisting}[caption={Partially decrypted text}, label={lst:partial_decryption}] 
heGGL?
whH aS D heYe? what's SH pVYpLse DB GDTe?

what PL D SeaB OH whL aS D?

MaGS PLwB, Eet  a  EYDp  BLw  ...  Lh!  thDs  Ds  aB  DBteYestDBE
seBsatDLB,  what  Ds  Dt?  Dt's  a  sLYt LT ... HawBDBE, tDBEGDBE
seBsatDLB DB SH ... SH  ...  weGG  D  sVppLse  D'P  OetteY  staYt
TDBPDBE  BaSes  TLY  thDBEs DT D waBt tL SaNe aBH heaPwaH DB what
TLY the saNe LT what D shaGG MaGG aB aYEVSeBt D  shaGG  MaGG  the
wLYGP, sL Get's MaGG Dt SH stLSaMh.

ELLP. LLLLh, Dt's EettDBE ZVDte stYLBE.  aBP  heH,  what's  aOLVt
thDs  whDstGDBE  YLaYDBE sLVBP ELDBE past what D'S sVPPeBGH ELDBE
tL MaGG SH heaP? peYhaps D MaB MaGG that ... wDBP! Ds that a ELLP
BaSe?  Dt'GG PL ... peYhaps D MaB TDBP a OetteY BaSe TLY Dt GateY
wheB D'Qe TLVBP LVt what Dt's TLY.  Dt  SVst  Oe  sLSethDBE  QeYH
DSpLYtaBt  OeMaVse theYe MeYtaDBGH seeSs tL Oe a heGG LT a GLt LT
Dt. heH! what's thDs thDBE? thDs ... Get's MaGG Dt a taDG - Heah,
taDG.  heH! D MaB MaB YeaGGH thYash Dt aOLVt pYettH ELLP MaB't D?
wLw! wLw! that TeeGs EYeat! PLesB't seeS tL aMhDeQe QeYH SVMh OVt
D'GG pYLOaOGH TDBP LVt what Dt's TLY GateY LB. BLw - haQe D OVDGt
Vp aBH MLheYeBt pDMtVYe LT thDBEs Het?

BL. BeQeY SDBP, heH, thDs Ds YeaGGH eJMDtDBE, sL  SVMh  tL  TDBP  LVt
aOLVt,  sL  SVMh  tL  GLLN  TLYwaYP  tL,  D'S  ZVDte  PDFFH  wDth
aBtDMDpatDLB ... LY Ds Dt the wDBP?

theYe YeaGGH Ds a GLt LT that BLw DsB't Dt?

aBP wLw! heH! what's thDs thDBE sVPPeBGH MLSDBE tLwaYPs  Se  QeYH
Tast?  QeYH  QeYH Tast. sL ODE aBP TGat aBP YLVBP, Dt BeePs a ODE
wDPe sLVBPDBE BaSe GDNe ... Lw ... LVBP  ...  YLVBP  ...  EYLVBP!
that's Dt! that's a ELLP BaSe - EYLVBP!

D wLBPeY DT Dt wDGG Oe TYDeBPs wDth Se?

aBP the Yest, aTteY a sVPPeB wet thVP, was sDGeBMe.

MVYDLVsGH eBLVEh, the LBGH thDBE that weBt thYLVEh  the  SDBP  LT
the OLwG LT petVBDas as Dt TeGG was Lh BL, BLt aEaDB. SaBH peLpGe
haQe speMVGateP that DT we NBew eJaMtGH whH the OLwG LT  petVBDas
haP thLVEht that we wLVGP NBLw a GLt SLYe aOLVt the BatVYe LT the
VBDQeYse thaB we PL BLw.

sLVYMe: http://www.MGeaYwhDteGDEht.LYE/hDtMh/hhEttE.tJt
\end{lstlisting}
Looking through the text, cipher letter 'D' is the only letter that is alone, which implied that this was the plaintext letter 'I'.
The first word in the text also stands out. 
In the beginning it was uncertain what it could be, but given that it had to repetitive letters (cipher letter 'G'), this could mean that this would in plain text be "hello".
It now also become more obvious that the word before the ciphered link had to be the word "source".
Further investigation when looking at the link, and this partially decrypted sentence ("\ldots Eet  a  Erip  Bow \ldots"), the cipher text 'E' was equal to plaintext 'G'.

\begin{lstlisting}[caption={Partially decrypted text - part 2}, label={lst:partial_decryption2}] 
hello?
whH aS i here? what's SH purpose iB liTe?

what Po i SeaB OH who aS i?

calS PowB, get  a  grip  Bow  ...  oh!  this  is  aB  iBterestiBg
seBsatioB,  what  is  it?  it's  a  sort oT ... HawBiBg, tiBgliBg
seBsatioB iB SH ... SH  ...  well  i  suppose  i'P  Oetter  start
TiBPiBg  BaSes  Tor  thiBgs iT i waBt to SaNe aBH heaPwaH iB what
Tor the saNe oT what i shall call aB arguSeBt i  shall  call  the
worlP, so let's call it SH stoSach.

gooP. ooooh, it's gettiBg Zuite stroBg.  aBP  heH,  what's  aOout
this  whistliBg  roariBg souBP goiBg past what i'S suPPeBlH goiBg
to call SH heaP? perhaps i caB call that ... wiBP! is that a gooP
BaSe?  it'll Po ... perhaps i caB TiBP a Oetter BaSe Tor it later
wheB i'Qe TouBP out what it's Tor.  it  Sust  Oe  soSethiBg  QerH
iSportaBt  Oecause there certaiBlH seeSs to Oe a hell oT a lot oT
it. heH! what's this thiBg? this ... let's call it a tail - Heah,
tail.  heH! i caB caB reallH thrash it aOout prettH gooP caB't i?
wow! wow! that Teels great! PoesB't seeS to achieQe QerH Such Out
i'll proOaOlH TiBP out what it's Tor later oB. Bow - haQe i Ouilt
up aBH cohereBt picture oT thiBgs Het?

Bo. BeQer SiBP, heH, this is reallH eJcitiBg, so  Such  to  TiBP  out
aOout,  so  Such  to  looN  TorwarP  to,  i'S  Zuite  PiFFH  with
aBticipatioB ... or is it the wiBP?

there reallH is a lot oT that Bow isB't it?

aBP wow! heH! what's this thiBg suPPeBlH coSiBg towarPs  Se  QerH
Tast?  QerH  QerH Tast. so Oig aBP Tlat aBP rouBP, it BeePs a Oig
wiPe souBPiBg BaSe liNe ... ow ... ouBP  ...  rouBP  ...  grouBP!
that's it! that's a gooP BaSe - grouBP!

i woBPer iT it will Oe TrieBPs with Se?

aBP the rest, aTter a suPPeB wet thuP, was sileBce.

curiouslH eBough, the oBlH thiBg that weBt through  the  SiBP  oT
the Oowl oT petuBias as it Tell was oh Bo, Bot agaiB. SaBH people
haQe speculateP that iT we NBew eJactlH whH the Oowl oT  petuBias
haP thought that we woulP NBow a lot Sore aOout the Bature oT the
uBiQerse thaB we Po Bow.

source: http://www.clearwhitelight.org/hitch/hhgttg.tJt
\end{lstlisting}
After getting this much of the text decrypted, the rest of the mapping was done based on what letters was most probable for the given word. 
The resulting mapping table is shown below.
\begin{center}
	\begin{tabular}{ | c | c | c | } \hline 
		A$_{plaintext}$ & $\rightarrow$ & U$_{ciphertext}$ \\ \hline 
		A & $\rightarrow$ & C \\ \hline 
		B & $\rightarrow$ & O \\ \hline 
		C & $\rightarrow$ & M \\ \hline 
		D & $\rightarrow$ & P \\ \hline 
		E & $\rightarrow$ & U \\ \hline 
		F & $\rightarrow$ & T \\ \hline 
		G & $\rightarrow$ & E \\ \hline 
		H & $\rightarrow$ & R \\ \hline 
		I & $\rightarrow$ & D \\ \hline
		J & $\rightarrow$ & I \\ \hline 
		K & $\rightarrow$ & N \\ \hline 
		L & $\rightarrow$ & G \\ \hline 
		M & $\rightarrow$ & S \\ \hline 
		N & $\rightarrow$ & B \\ \hline 
		O & $\rightarrow$ & L \\ \hline 
		P & $\rightarrow$ & A \\ \hline 
		Q & $\rightarrow$ & Z \\ \hline 
		R & $\rightarrow$ & Y \\ \hline 
		S & $\rightarrow$ & X \\ \hline 
		T & $\rightarrow$ & W \\ \hline 
		U & $\rightarrow$ & V \\ \hline 
		V & $\rightarrow$ & Q \\ \hline 
		W & $\rightarrow$ & K \\ \hline 
		X & $\rightarrow$ & J \\ \hline 
		Y & $\rightarrow$ & H \\ \hline 
		Z & $\rightarrow$ & F \\ \hline 
	\end{tabular}
\end{center}

Quote:
\begin{quote} 
"hello?
why am i here? what's my purpose in life?

what do i mean by who am i?

calm down, get  a  grip  now  ...  oh!  this  is  an  interesting
sensation,  what  is  it?  it's  a  sort of ... yawning, tingling
sensation in my ... my  ...  well  i  suppose  i'd  better  start
finding  names  for  things if i want to make any headway in what
for the sake of what i shall call an argument i  shall  call  the
world, so let's call it my stomach."
\end{quote}

\subsection{1d) Key space}
\label{section:1d}
\begin{comment}
What is the key space of this substitution cipher, with an alphabet consisting of 26 letters?
\end{comment}
The key space for 26 letters is 26! {\raise.17ex\hbox{$\scriptstyle\sim$}}= 2$^{88}$.

\subsection{1e) Book and Author}
\label{section:1e}
Based on the link in the cipher-text (\url{http://www.clearwhitelight.org/hitch/hhgttg.txt}), the book is "The Hitchhiker's Guide to the Galaxy".
The author is Douglas N. Adams.

\section{Modulare Arithmetik I}
\label{section:mod_arithmetik1}

\subsection{5 * 9 mod 19}
\label{section:2a}
\begin{equation}
	\begin{split}
		5 \times 9 \mod 19 \\
		\Rightarrow 45 \mod 19 \\
		\Rightarrow \frac{45}{19} \\
		\Rightarrow 19 \times 2 = 38; \\
		45 - 38 = 7 \\
		\Rightarrow 45 \mod 19 = 7 
	\end{split}
\end{equation}

\begin{comment}
\subsection{24 * 47 mod 19}
\label{section:2b}
Note: Fractioning was used in the second step to show how 24 * 47 was calculated. 
When multiplying with numbers that both were equal to or greater than 11, we were taught that one first multiples the last number with all numbers in the first value. 
Thereafter, you add these numbers by moving the next number one spot forward, indicating that the next number is now being multiplied. 
This is why I have in the equation below added a '0' at the end of the first fractioning (this fractioning is here not a fraction, but a representation of this multiplication).
\begin{equation}
	\begin{split}
		24 \times 47 \mod 19 \\
		\rightarrow 24 \times 47 = \frac{~~~168}{+~960}  \\
		\Rightarrow 1128 \mod 19  \\
		\Rightarrow \frac{1128}{19}
	\end{split}
\end{equation}

\subsection{(-33) * (-11) mod 19}
\label{section:2c}
\begin{equation}
	\begin{split}
		(-33) \times (-11) \mod 19 \\
		(negative \times negative = positive) \\
		\rightarrow 33 \times 11 = \frac{~~~033}{+~330}  \\
		\Rightarrow 363 \mod 19 \\
		\Rightarrow \frac{363}{19}
	\end{split}
\end{equation}

\subsection{(-14) * 28 mod 19}
\label{section:2d}
\begin{equation}
	\begin{split}
		(-14) \times 28 \mod 19 \\
		\rightarrow (-14) \times 28 = -(\frac{~~~112}{+~280})  \\
		\Rightarrow -392 \mod 19 \\
		\Rightarrow \frac{-392}{19}
	\end{split}
\end{equation}

\section{Modulare Arithmetik II}
\label{section:mod_arithmetik2}

\subsection[1/9 mod 17 = 9\textsuperscript{-1} mod 17]{$\frac{1}{9} \mod 17 = 9^{-1} \mod 17$}
\label{section:3a}
\begin{equation}
	\begin{split}
		\frac{1}{9} \mod 17 = 9^{-1} \mod 17 
	\end{split}
\end{equation}

\subsection[2/8 mod 23 = 2 * 8\textsuperscript{-1} mod 23]{$\frac{2}{8} \mod 23 = 2 \times 8^{-1} \mod 23$}
\label{section:3b}
\begin{equation}
	\begin{split}
		\frac{2}{8} \mod 23 = 2 \times 8^{-1} \mod 23
	\end{split}
\end{equation}

\subsection[4/7 mod 13 = 4 * 7\textsuperscript{-1} mod 13]{$\frac{4}{7} \mod 13 = 9^{-1} \mod 13$}
\label{section:3c}
\begin{equation}
	\begin{split}
		\frac{4}{7} \mod 13 = 9^{-1} \mod 13
	\end{split}
\end{equation}

\end{comment}

\section{Cäsar-Chiffre}
\label{section:caesar_cipher}

\subsection{Alphabet mapping}
\label{section:4a}
When shifting letters, you just move N letters forward into the alphabet. 
If the offset is 6, then 'A'$\Rightarrow$ 'G', 'B' $\Rightarrow$ 'H', etc.
Which means that for the letters 'V' to 'Z' you get the following table:

\begin{tabular}{ | c | c | c | c | c | c | c | c | c | } \hline 
	Plain letter	& A & B & \ldots & V & W & X & Y & Z \\ \hline 
	Shifted letter 	& G & H & \ldots & B & C & D & E & F \\ \hline 
\end{tabular}

\subsection{Decrypt "pelcgbtencul"}
\label{section:4b}
The encrypted ciphertext "pelcgbtencul" with an offset with k=13 becomes the text "CRYPTOGRAPHY"\footnote{
	Decrypted by using "shift\_letters": \\ 
	\url{https://github.com/klAndersen/Msc-Appl-Info-Sec/tree/master/Module\%201\%20-\%20Intro\%20to\%20Crypto/Uebung_01}
}. This shift is known as ROT13.

\subsection{Using offset k=26}
\label{section:4c}
The use of the offset k=26 is not practical, because the offset is based on the amount of letters you shift. 
Since the alphabet contains 26 letters, you would end up going from 'A' + 26 letters $\Rightarrow$ 'A'.
You basically end up with a cipher text that is equal to the plaintext. 

\subsection{Statistical attacks}
\label{section:4d}
\begin{comment}
d) When you compare the security of the Caesar cipher to the substitution cipher with respect to statistic attacks, which one is more secure? 
\end{comment}
Substitution cipher is more secure, because it uses a key of varying length. 
When it comes to Caesar cipher, it only uses an offset. 
Therefore, if you use a statistical attack and find that a given cipher is repeated often, one can assume that this letter is either an 'E' or 'T' (presuming English text).
Then one could simply map the 'E' based on its belonging cipher,map the remaining letter based on the offset, and see if the plaintext "makes sense". 
