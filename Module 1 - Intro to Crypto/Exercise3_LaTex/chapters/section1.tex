\section{Addition in GF(2$^{8}$)}

%P(x) = x$^{8}$ + x$^{4}$ + x$^{3}$ + x + 1.

\subsection{A(x) = x$^{7}$ + x$^{5}$ + x$^{3}$ + x$^{2}$ + 1, B(x) = x$^{4}$ + x$^{3}$ + 1}
\label{section:1a}
Since addition uses XOR, I have "padded" the equation with zero's to better display the difference (and equality) between A(x) and B(x).
For this, the \LaTeX \textit{\\frac\{num\}\{den\}} was used.
\begin{equation}
\begin{split}
A(x) = x^{7} + x^{5} + x^{3} + x^{2} + 1, \\
B(x) = x^{4} + x^{3} + 1 \\
~ \\
\textit{Comparison listing:} \\
+ \frac{A(x)}{B(x)} \Rightarrow + \frac{x^{7} + 0 + x^{5} + 0 + x^{3} + x^{2} + 0 + 1}{0 + 0 + 0 + x^{4} + x^{3} + 0 + 0 + 1} \\
~ \\
\textit{Modulus 2 cancels out those that are equal:} \\
A (x) + B(x) \Rightarrow x^{7} + x^{5} + x^{4} + (1 + 1) x^{3} + x^{2} + (1 + 1) \\
A (x) + B(x) \Rightarrow x^{7} + x^{5} + x^{4} + 0x^{3} + x^{2} + 0 \\
\underline{\underline{A (x) + B(x) = x^{7} + x^{5} + x^{4} + x^{2}}}
\end{split}
\end{equation}

\subsection{A(x) = x$^{5}$ +x$^{3}$ + x$^{2}$, B(x) = x$^{6}$ + x$^{4}$ + x$^{3}$ + 1}
\label{section:1b}
\begin{equation}
\begin{split}
A(x) = x^{5} + x^{3} + x^{2}, \\
B(x) = x^{6} + x^{4} + x^{3} + 1  \\
~ \\
\textit{Comparison listing:} \\
+ \frac{A(x)}{B(x)} \Rightarrow + \frac{0 + x^{5} + 0 + x^{3} + x^{2} + 0 + 0}{x^{6} + 0 + x^{4} + x^{3} +0 + 0 + 1} \\
~ \\
\textit{Modulus 2 cancels out those that are equal:} \\
A (x) + B(x) \Rightarrow x^{6} + x^{5} + x^{4} + (1 + 1)x^{3} + x^{2} + 1 \\
A (x) + B(x) \Rightarrow x^{6} + x^{5} + x^{4} + 0x^{3} + x^{2} + 1 \\
\underline{\underline{A (x) + B(x) = x^{6} + x^{5} + x^{4} + x^{2} + 1}}
\end{split}
\end{equation}

\subsection{A(x) = x$^{6}$ + x$^{5}$ + x$^{3}$ + x + 1, B(x) = x$^{7}$ + x$^{6}$ + x$^{4}$ + x$^{2}$ + 1}
\label{section:1c}
\begin{equation}
\begin{split}
A(x) = x^{6} + x^{5} + x^{3} + x + 1, \\
B(x) = x^{7} + x^{6} + x^{4} + x^{2} + 1
~ \\
\textit{Comparison listing:} \\
+ \frac{A(x)}{B(x)} \Rightarrow + \frac{0 + x^{6} + x^{5} + 0 + x^{3} + 0 + x + 1}{x^{7} + x^{6} + 0 + x^{4} + 0 + x^{2} + 0 + 1} \\
~ \\
\textit{Modulus 2 cancels out those that are equal:} \\
A (x) + B(x) \Rightarrow x^{7} + (1 + 1)x^{6} + x^{5} + x^{4} +  x^{3} + x^{2} + x + (1 + 1)  \\
A (x) + B(x) \Rightarrow x^{7} + 0x^{6} + x^{5} + x^{4} +  x^{3} + x^{2} + x + 0 \\
\underline{\underline{A (x) + B(x) = x^{7} + x^{5} + x^{4} +  x^{3} + x^{2} + x}}
\end{split}
\end{equation}

\subsection{Which effect has the reduction polynome in general on the result of an addition?}
\label{section:1d}
The reduction polynome in general has a XOR effect both for addition and substitution. 
This means you can just use the XOR to find the result, because each number is its additive inverse.

\section{Multiplication in GF(5$^{4}$)}
Consider the finite field F(5$^{4}$) with the irreducible reduction polynome P(x) = x$^{4}$ + x$^{2}$ + 2x + 2.

\subsection{Compute the addition table for this field}
\label{section:2a}
% this means a n  n-table, with the results of the addition of the x- and the y-coordinate (0 to n-1).

\subsection{Compute the multiplication table for this field.}
\label{section:2b}

\subsection{Compute x$^{4}$ mod P(x), x$^{5}$ mod P(x) and x$^{6}$ mod P(x).}
\label{section:2c}
\begin{equation}
\begin{split}
x^{4} mod P(x), x^{5} mod P(x) \\ 
x^{6} mod P(x)
\end{split}
\end{equation}

\subsection{Calculate A(x) $\times$ B(x) mod P(x) for A(x) = x$^{4}$ + x$^{1}$ + 2, B(x) = 2x$^{3}$ + 2x$^{2}$ + 1}
\label{section:2d}
\begin{equation}
\begin{split}
A(x) \times B(x)~mod~P(x); \\
A(x) = x^{4} + x^{1} + 2, \\ 
B(x) = 2x^{3} + 2x^{2} + 1
\end{split}
\end{equation}

\section{Multiplication in GF(2$^{8}$)}

\subsection{Compute A(x) $\times$ B(x) mod P(x) for the following values and give the result in HEX}
\label{section:3a}

\subsubsection{A(x) = x$^{7}$ + x$^{4}$ + x$^{x3}$ + x + 1, B(x) = x }
\label{section:3aa}
\begin{equation}
\begin{split}
A(x) = x^{7} + x^{4} + x^{x3} + x + 1, \\
B(x) = x 
\end{split}
\end{equation}

\subsubsection{x$^{6}$ + x$^{3}$ + x + 1, B(x) = x + 1}
\label{section:3ab}
\begin{equation}
\begin{split}
A(x) = x^{6} + x^{3} + x + 1,  \\
B(x) = x + 1 
\end{split}
\end{equation}

\subsubsection{x$^{7}$ + x$^{6}$ + x$^{5}$, B(x) = x$^{3}$ + x}
\label{section:3ac}
\begin{equation}
\begin{split}
A(x) = x^{7} + x^{6} + x^{5}, \\ 
B(x) = x^{3} + x
\end{split}
\end{equation}

\subsection{With which operation is it possible to realise both these multiplications B$_{1}$(x) = x, B$_{2}$(x) = x+1 efficiently}
\label{section:3b}

\section{Avalanche effect in AES}
\begin{comment}
W = ($w_{0};w_{1};w_{2};w_{3}$) = (0x00000000; 0x00100000; 0x00000000; 0x00000000)

K0 = (0x2B7E1516)
(0x28AED2A6)
(0xABF71588)
(0x09CF4F3C)

K1 = (0xA0FAFE17)
(0x88542CB1)
(0x23A33939)
(0x2A6C7605)
\end{comment}

\subsection{Calculate the respective Output to the Input W after the first round of AES!}
\label{section:4a}
%  use the round-keys K$_{0}$, \ldots, K$_{1}$!

\subsection{Compute all the output bytes for the case that all the input bytes are zero (solution only in HEX)}
\label{section:4b}

\subsection{How many outputbytes have changed now? (We consider just one round of AES)}
\label{section:4c}

\section{Keygeneration in AES}

\subsection{Given is a main key K, consisting of zeros. Find the sub-key K$_{1}$ after the first round of key-generation.}
\label{section:5a}

\subsection{Given is the main key K = (0x00000008; 0x00000004; 0x00000002; 0x00000001). Find the sub-key K$_{1}$ after the first round of key-generation.}
\label{section:5b}
% K = (0x00000008; 0x00000004; 0x00000002; 0x00000001).

\section{Solution template for Avalanche effect in AES}

\subsection{After conversion of the Input in matrix-form}
\label{section:6a}

\subsection{After conversion of the Input in matrix-form}
\label{section:6b}
