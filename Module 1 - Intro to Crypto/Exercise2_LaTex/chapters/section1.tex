\section{DES-history}

\subsection{What was/is the name of the authority that called out for the development of DES?}
\label{section:1a}

\subsection{In which year DES was standardised?}
\label{section:1b}

\subsection{Which authority also was said to have influenced the standardisation of DES?}
\label{section:1c}

\subsection{Which company did submit the cipher?}
\label{section:1d}

\subsection{Which kind of structure is used in Lucifer?}
\label{section:1e}

\subsection{Which key length was supported by Lucifer, originally?}
\label{section:1f}


\section{Basics of block ciphers}

\subsection{what meaning does the (balanced) feistel-network have for the processing of data?}
\label{section:2a}

\subsection{Point out the characteristics of a feistel-structure related to the encryption and decryption.}
\label{section:2b}

\subsection{Claude Shannon says there are two types of primitive Operations so build up strong encryption algorithms. Name and describe these two in a few sentences.}
\label{section:2c}

\section{Avalanche effect in DES}

\begin{comment}
For a good block cipher it is important that a small change in the input bits results in a great change
in the output bits( This effect is called avalanche effect). In the following we will try to evaluate this
effect in DES. For this we use an input string with an one at the 19th position(x25 = 1). All other
Bits are set to zeros. The 56 bits of the key also are set to zero(this means all the round keys are zero.
Remember that the input bit string first runs through the initial permutation!)
\end{comment}

\subsection{Which S-boxes are influenced by this bit in the first round. Also calculate the input bits of all S-boxes.}
\label{section:3a}

\subsection{Calculate the bit string at the end of the first round. (L1 und R1)}
\label{section:3b}

\subsection{Calculate the output bits for the case that all input bits are zeros! (x25=0). How many bits did change in L1 and R1 compared to exercise b)?}
\label{section:3c}


\section{Non-linearity of S-boxes}

\begin{comment}
An important property of DES is the non-linearity of the S-boxes. In this exercise we want to verify
this property by comparing the output bits for different input bits in a specific S-box S i. Show that for
S 5 the following computation rule can be applied: 2
S i(x1) bXOR S i(x2) , S i(x1 bXOR x2):
"bXOR" is called a bitwise XOR.
\end{comment}

\subsection{x1 = 000001, x2 = 100000}
\label{section:4a}

\subsection{x1 = 001100, x2 = 111001}
\label{section:4b}

\subsection{x1 = 010011, x2 = 011110}
\label{section:4c}


\section{Brute-Force Attacke auf den DES}

\begin{comment}
An often used method to evaluate the security of an symmetric encryption algorithm against a bruteforce
attack is the calculation of the costs of constructing a key search machine. In the next exercise
we will have a look at this thematic.
Consider a DES-chip with internal parallel hardware structure that can perform a DES-encryption in
a single cycle(So it can test one key per cycle). We further assume that we can run the chips with a
frequency of 126 MHz.
\end{comment}

\subsection{How many chips of this kind do we have to run parallel so we can calculate the DES-Key in a single day?}
\label{section:5a}

\subsection{How much would these chips cost if one chip costs 10 Euro and we calculate 100\% overhead for running the chips parallel, the power supply and anything else?}
	\label{section:5b}
	
\subsection{Why is this design of such a key searching machine only the upper limit of security?}
\label{section:5c}


\section{DES bit complement}

\begin{comment}
DES has an incredible property concerning the bitwise complement of the input and output bits. The
complement (meaning all Bits are inverted) of a bit stream A is given with A0(example: If A = 0110
the complement is A0 = 1001.) “bXOR“ is the bitwise XOR. We want to show that: If
y = DES_k(x)
than also is:
y_0 = DES_k0 (x0):
Meaning if we sent the complement of the message and the key than the complement of the original
cipher text is returned. Your exercise is to proof this property. (Hint: Try to proof the property for any
round instead of calculating all 16 rounds!)
\end{comment}
