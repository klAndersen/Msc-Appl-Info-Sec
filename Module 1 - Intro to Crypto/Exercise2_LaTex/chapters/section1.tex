\section{DES-history}

\subsection{What was/is the name of the authority that called out for the development of DES?}
\label{section:1a}
DES was developed by IBM in cooperation with NSA, because the US government asked for a standard. 
The representative for the US Goverment asking for the standard was NIST (National Institute of Standards and Technology), 
who at  the time was known as NBS (National Bureau of Standards) \cite[p.~56]{Paar2010}.

\subsection{In which year DES was standardised?}
\label{section:1b}
DES was first standardised in 1977, but it was only standardised for 10 years (up until 1987) \cite[p.~56]{Paar2010}.

\subsection{Which authority also was said to have influenced the standardisation of DES?}
\label{section:1c}
NSA was the authority said to have influenced the DES standard.

\subsection{Which company did submit the cipher?}
\label{section:1d}
IBM was the company that submitted the cipher, which was based on the Lucifer cipher. 
IBM originally named it Lucifer, but after the review from NSA a modified version was returned named DES \cite[p.~113]{Trappe2006}.

\subsection{Which kind of structure is used in Lucifer?}
\label{section:1e}
The structure used in Lucifer is the Feistel system (or Feistel Network \cite[p.~58]{Paar2010}), named after Horst Feistel who was part of the Lucifer development team \cite[p.~114]{Trappe2006}.

\subsection{Which key length was supported by Lucifer, originally?}
\label{section:1f}
The original key length supported by Lucifer was 128-bits \cite[p.~276-277]{Menezes1997}, \cite[p.~56]{Paar2010}.

\section{Basics of block ciphers}

\subsection{What meaning does the (balanced) Feistel-network have for the processing of data?}
\label{section:2a}
It shortens the processing time by only doing the mathematical operation on half the input. 
The other half is passed into the next round where it then is processed in the same way.

\subsection{Point out the characteristics of a Feistel-structure related to the encryption and decryption.}
\label{section:2b}
The main characteristic is that encryption and decryption are almost the same, and uses the same key for both operations. 
When decrypting, you are simply reversing the encryption process, where the only difference is that for decryption you need a reversed key schedule.

\subsection{Claude Shannon says there are two types of primitive operations to build up strong encryption algorithms.}
\label{section:2c}
The two types of primitive operations named by Shannon is Confusion and Diffusion. 
The aim of Confusion is to obscure the relationship between the plain - and cipher-text (e.g. substitution).
Diffusion focuses on hiding the plain-texts statistical properties, by spreading each plain-text bit over many cipher-text bits (e.g. permutation).
%p.57

\section{Avalanche effect in DES}
\subsection{Which S-boxes are influenced by this bit in the first round. Also calculate the input bits of all S-boxes.}
\label{section:3a}
Input-String: \{x$_{1}$ = 0, \ldots, x$_{19}$=1, \ldots, x$_{64}$ = 0\} \\
56-bit key: \{k$_{1}$ = 0, \ldots, k$_{56}$ = 0\} \\
IP-String: \{x$_{1}$ = 0, \ldots, x$_{45}$=1, \ldots, x$_{64}$ = 0\} \\
\vspace{1em}\newline
DES contains 16 rounds after the initial permutation. 
Through each round, the bits are split into two; left and right side. 
The right side is passed directly into the next rounds left side, but the current left side is encrypted by XOR'ing the content with the f-function.
The f-function contains 4 steps:
\begin{enumerate}
	\item Expansion: Using an expansion box (E-Box), going from 32-bits to 48-bits
	\item XOR with the 48-bits round key
	\item S-box substitution
	\item Permutation (to go back from 48-bits to 32-bits)
\end{enumerate}

\noindent
The right side consists of 32-bits, where the 13$^{th}$ bit is set to 1.
Based on the E-box shown in \cite[p.~63]{Paar2010}, the E-box looks like this:
E-Box: \{x$_{1}$ = 0, \ldots, x$_{18}$=1, x$_{19}$ = 0, x$_{20}$=1, \ldots, x$_{48}$ = 0\}. 
XOR'ing the E-box with the round key makes no changes (because 0 $\oplus$ 0 = 0; 1 $\oplus$ 1 = 0).
~
\vspace{1em}\newline
\noindent
Which S-box(es) are affected by the set bit (marked with 'X') and how do they look like? \\
Each S-box contains 6-bits, where the 6-bits comes from the result of the E-box $\oplus$ Round-key$_{i=1}$. This means that the only "affected" S-boxes are S3 and S4. 
\vspace{1em}\newline
\noindent
S1
\framebox[15pt][0]{}\quad
S2
\framebox[15pt][0]{}\quad
S3
\framebox[1.1\width]{X}\quad
S4
\framebox[1.1\width]{X}\quad
S5
\framebox[15pt][0]{}\quad
S6
\framebox[15pt][0]{}\quad
S7
\framebox[15pt][0]{}\quad
S8
\framebox[15pt][0]{}\quad
~
\vspace{1em}\newline
\noindent
S-Box 1: 14 = 1110\footnote{Based on S-box tables shown in \cite[p.~64-65]{Paar2010}.} \\
\begin{tabular}{ | l | l | l | l | l | l | } \hline 
	0 & 0 & 0 & 0 & 0 &  0 \\ \hline
\end{tabular}
~
\vspace{1em}\newline
\noindent
S-Box 2: 15 = 1111 \\
\begin{tabular}{ | l | l | l | l | l | l | } \hline 
0 & 0 & 0 & 0 & 0 &  0 \\ \hline
\end{tabular}
~
\vspace{1em}\newline
\noindent
S-Box 3: 13 = 1101 \\
\begin{tabular}{ | l | l | l | l | l | l | } \hline 
	0 & 0 & 0 & 0 & 0 &  1 \\ \hline
\end{tabular}
~
\vspace{1em}\newline
\noindent
S-Box 4: 01 = 0001 \\
\begin{tabular}{ | l | l | l | l | l | l | } \hline 
	0 & 1 & 0 & 0 & 0 &  0 \\ \hline
\end{tabular}
~
\vspace{1em}\newline
\noindent
S-Box 5: 02 = 0010 \\
\begin{tabular}{ | l | l | l | l | l | l | } \hline 
	0 & 0 & 0 & 0 & 0 &  0 \\ \hline
\end{tabular}
~
\vspace{1em}\newline
\noindent
S-Box 6: 12 = 1100 \\
\begin{tabular}{ | l | l | l | l | l | l | } \hline 
	0 & 0 & 0 & 0 & 0 &  0 \\ \hline
\end{tabular}
~
\vspace{1em}\newline
\noindent
S-Box 7: 04 = 0100 \\
\begin{tabular}{ | l | l | l | l | l | l | } \hline 
	0 & 0 & 0 & 0 & 0 &  0 \\ \hline
\end{tabular}
~
\vspace{1em}\newline
\noindent
S-Box 8: 13 = 1101 \\
\begin{tabular}{ | l | l | l | l | l | l | } \hline 
	0 & 0 & 0 & 0 & 0 &  0 \\ \hline
\end{tabular}
~
\vspace{0.5em}\newline
\noindent
S-box output: \\
\begin{tabular}{ | l | l | l | l | l | l | l |  l | } \hline 
	1 & 1 & 1 & 0 & 1 &  1 & 1 & 1 \\ \hline
	1 & 1 & 0 & 1 & 0 & 0 & 0 & 1 \\ \hline
	0 & 0 & 1 & 0 & 1 &  1 & 0 & 0 \\ \hline
	0 & 1 & 0 & 0 & 1 &  1 & 0 & 1 \\ \hline
\end{tabular}
~
\vspace{0.5em}\newline
\noindent
After permutation on the outputted S-boxes\footnote{Based on permutation table shown in \cite[p.~66]{Paar2010}.}: \\
\begin{tabular}{ | l | l | l | l | l | l | l |  l | } \hline 
	1 & 1 & 0 & 1 & 1 &  1 & 0 & 0 \\ \hline
	1 & 0 & 0 & 1 & 1 &  0 & 0 & 1 \\ \hline
	1 & 1 & 0 & 0 & 1 &  0 & 1 & 0 \\ \hline
	1 & 0 & 1 & 1 & 1 &  1 & 0 & 0 \\ \hline
\end{tabular}

\subsection{Calculate the bit string at the end of the first round. (L1 and R1)}
\label{section:3b}
L$_{1}$ = R$_{0}$ \\
\begin{tabular}{ | l | l | l | l | l | l | l |  l |  } \hline 
	0 & 0 & 0 & 0 & 0 &  0 & 0 & 0 \\ \hline
	0 & 0 & 0 & 0 & 1 &  0 & 0 & 0 \\ \hline
	0 & 0 & 0 & 0 & 0 &  0 & 0 & 0 \\ \hline
	0 & 0 & 0 & 0 & 0 &  0 & 0 & 0 \\ \hline
\end{tabular}
\clearpage\noindent
% space
R$_{1}$ = L$_{1} \oplus$ f(R$_{0} \oplus K_{0}$)\\
\begin{tabular}{ | l | l | l | l | l | l | l |  l |  } \hline 
	1 & 1 & 0 & 1 & 1 &  1 & 0 & 0 \\ \hline
	1 & 0 & 0 & 1 & 0 &  0 & 0 & 1 \\ \hline
	1 & 1 & 0 & 0 & 1 &  0 & 1 & 0 \\ \hline
	1 & 0 & 1 & 1 & 1 &  1 & 0 & 0 \\ \hline
\end{tabular}

\subsection{Calculate the output bits for the case that all input bits are zeros!  How many bits did change in L1 and R1 compared to exercise b)?}
\label{section:3c}
Input-String: \{x$_{1}$ = 0, \ldots, x$_{64}$ = 0\} \\
56-bit key: \{k$_{1}$ = 0, \ldots, k$_{56}$ = 0\} \\
IP-String: \{x$_{1}$ = 0, \ldots, x$_{64}$ = 0\} 
~
\vspace{1em}\newline
\noindent
S-Box 1: 14 = 1110\footnote{Based on S-box tables shown in \cite[p.~64-65]{Paar2010}.} \\
\begin{tabular}{ | l | l | l | l | l | l | } \hline 
	0 & 0 & 0 & 0 & 0 &  0 \\ \hline
\end{tabular}
~
\vspace{1em}\newline
\noindent
S-Box 2: 15 = 1111 \\
\begin{tabular}{ | l | l | l | l | l | l | } \hline 
	0 & 0 & 0 & 0 & 0 &  0 \\ \hline
\end{tabular}
~
\vspace{1em}\newline
\noindent
S-Box 3: 10 = 1010 \\
\begin{tabular}{ | l | l | l | l | l | l | } \hline 
	0 & 0 & 0 & 0 & 0 &  0 \\ \hline
\end{tabular}
~
\vspace{1em}\newline
\noindent
S-Box 4: 07 = 0111 \\
\begin{tabular}{ | l | l | l | l | l | l | } \hline 
	0 & 0 & 0 & 0 & 0 &  0 \\ \hline
\end{tabular}
~
\vspace{1em}\newline
\noindent
S-Box 5: 02 = 0010 \\
\begin{tabular}{ | l | l | l | l | l | l | } \hline 
	0 & 0 & 0 & 0 & 0 &  0 \\ \hline
\end{tabular}
~
\vspace{1em}\newline
\noindent
S-Box 6: 12 = 1100 \\
\begin{tabular}{ | l | l | l | l | l | l | } \hline 
	0 & 0 & 0 & 0 & 0 &  0 \\ \hline
\end{tabular}
~
\vspace{1em}\newline
\noindent
S-Box 7: 04 = 0100 \\
\begin{tabular}{ | l | l | l | l | l | l | } \hline 
	0 & 0 & 0 & 0 & 0 &  0 \\ \hline
\end{tabular}
~
\vspace{1em}\newline
\noindent
S-Box 8: 13 = 1101 \\
\begin{tabular}{ | l | l | l | l | l | l | } \hline 
	0 & 0 & 0 & 0 & 0 &  0 \\ \hline
\end{tabular}
~
\vspace{0.5em}\newline
\noindent
S-box output: \\
\begin{tabular}{ | l | l | l | l | l | l | l |  l | } \hline 
	1 & 1 & 1 & 0 & 1 &  1 & 1 & 1 \\ \hline
	1 & 0 & 1 & 0 & 0 &  1 & 1 & 1 \\ \hline
	0 & 0 & 1 & 0 & 1 &  1 & 0 & 0 \\ \hline
	0 & 1 & 0 & 0 & 1 &  1 & 0 & 1 \\ \hline
\end{tabular}
~
\vspace{0.5em}\newline
\noindent
After permutation on the outputted S-boxes\footnote{Based on permutation table shown in \cite[p.~66]{Paar2010}.}: \\
\begin{tabular}{ | l | l | l | l | l | l | l |  l | } \hline 
	1 & 1 & 0 & 1 & 1 &  0 & 0 & 0 \\ \hline
	1 & 1 & 0 & 1 & 1 &  0 & 0 & 0 \\ \hline
	1 & 1 & 0 & 1 & 1 &  0 & 1 & 1 \\ \hline
	1 & 0 & 1 & 1 & 1 &  1 & 0 & 0 \\ \hline
\end{tabular} \\
\noindent
L$_{1}$ = R$_{0}$ \\
\begin{tabular}{ | l | l | l | l | l | l | l |  l |} \hline
	0 & 0 & 0 & 0 & 0 &  0 & 0 & 0 \\ \hline
	0 & 0 & 0 & 0 & 0 &  0 & 0 & 0 \\ \hline
	0 & 0 & 0 & 0 & 0 &  0 & 0 & 0 \\ \hline
	0 & 0 & 0 & 0 & 0 &  0 & 0 & 0 \\ \hline
\end{tabular}
~
\vspace{0.5em}\newline\noindent
% space
R$_{1}$ = L$_{1} \oplus$ f(R$_{0} \oplus K_{0}$)\\
\begin{tabular}{ | l | l | l | l | l | l | l |  l |} \hline
	1 & 1 & 0 & 1 & 1 &  0 & 0 & 0 \\ \hline
	1 & 1 & 0 & 1 & 1 &  0 & 0 & 0 \\ \hline
	1 & 1 & 0 & 1 & 1 &  0 & 1 & 1 \\ \hline
	1 & 0 & 1 & 1 & 1 &  1 & 0 & 0 \\ \hline
\end{tabular}
~
\vspace{0.5em}\newline\noindent
% space	
Amount of flipped bits: 7. \\ 
1 in L$_{1}$ and 6 in R$_{1}$.

\section{Non-linearity of S-boxes}
S$_{i}$(x$_{1}$) $\oplus$ S$_{i}$(x$_{2}$) $\neq$ S$_{i}$(x$_{1} \oplus$ x$_{2}$)

\begin{comment}
An important property of DES is the non-linearity of the S-boxes. In this exercise we want to verify
this property by comparing the output bits for different input bits in a specific S-box S i. Show that for
S_5 the following computation rule can be applied: 2
S i(x1) bXOR S i(x2) , S i(x_1 bXOR x_2):
"bXOR" is called a bitwise XOR.
\end{comment}

\subsection{x1 = 000001, x2 = 100000}
\label{section:4a}
S$_{5}$(x$_{1}$) = S$_{5}$(000001) = 14 = 1110 \\
S$_{5}$(x$_{2}$) = S$_{5}$(100000) = 04 = 0100 
\vspace{0.5em}\newline
S$_{5}$(x$_{1}$) $\oplus$ S$_{5}$(x$_{2}$) = 1110 $\oplus$ 0100 = \underline{1010}
\vspace{1em}\newline
x$_{1} \oplus$ x$_{2}$ = 000001 $\oplus$ 100000 = 100001 \\
S$_{5}$(x$_{1} \oplus$ x$_{2}$) = S$_{5}$(100001) = 11 = \underline{1011}
\vspace{1em}\newline
\underline{\underline{S$_{5}$(x$_{1}$) $\oplus$ S$_{5}$(x$_{2}$) $\neq$ S$_{5}$(x$_{1} \oplus$ x$_{2}$) $\Rightarrow$ 1010  $\neq$ 1011}}

\subsection{x1 = 001100, x2 = 111001}
\label{section:4b}
S$_{5}$(x$_{1}$) = S$_{5}$(001100) = 11 = 1011 \\
S$_{5}$(x$_{2}$) = S$_{5}$(111001) = 10 = 1010 
\vspace{0.5em}\newline
S$_{5}$(x$_{1}$) $\oplus$ S$_{5}$(x$_{2}$) = 1011 $\oplus$ 1010 = \underline{0001}
\vspace{1em}\newline
x$_{1} \oplus$ x$_{2}$ = 001100 $\oplus$ 111001  = 110101 \\
S$_{5}$(x$_{1} \oplus$ x$_{2}$) = S$_{5}$(110101) = 00 = \underline{0000}
\vspace{1em}\newline
\underline{\underline{S$_{5}$(x$_{1}$) $\oplus$ S$_{5}$(x$_{2}$) $\neq$ S$_{5}$(x$_{1} \oplus$ x$_{2}$) $\Rightarrow$ 0001  $\neq$ 0000}}

\subsection{x1 = 010011, x2 = 011110}
\label{section:4c}
S$_{5}$(x$_{1}$) = S$_{5}$(010011) = 00 = 0000 \\
S$_{5}$(x$_{2}$) = S$_{5}$(011110) = 09 = 1001 
\vspace{0.5em}\newline
S$_{5}$(x$_{1}$) $\oplus$ S$_{5}$(x$_{2}$) = 0000 $\oplus$ 1001 = \underline{1001}
\vspace{1em}\newline
x$_{1} \oplus$ x$_{2}$ = 010011 $\oplus$ 011110  = 001101 \\
S$_{5}$(x$_{1} \oplus$ x$_{2}$) = S$_{5}$(001101) = 13 = \underline{1101}
\vspace{1em}\newline
\underline{\underline{S$_{5}$(x$_{1}$) $\oplus$ S$_{5}$(x$_{2}$) $\neq$ S$_{5}$(x$_{1} \oplus$ x$_{2}$) $\Rightarrow$ 1001  $\neq$ 1101}}

\section{Brute-Force Attacke auf den DES}

\begin{comment}
An often used method to evaluate the security of an symmetric encryption algorithm against a bruteforce
attack is the calculation of the costs of constructing a key search machine. In the next exercise
we will have a look at this thematic.
Consider a DES-chip with internal parallel hardware structure that can perform a DES-encryption in
a single cycle(So it can test one key per cycle). We further assume that we can run the chips with a
frequency of 126 MHz.
\end{comment}

\subsection{How many chips of this kind do we have to run parallel so we can calculate the DES-Key in a single day?}
\label{section:5a}
- 

\subsection{How much would these chips cost if one chip costs 10 Euro and we calculate 100\% overhead for running the chips parallel, the power supply and anything else?}
\label{section:5b}
-
\subsection{Why is this design of such a key searching machine only the upper limit of security?}
\label{section:5c}
Because:
\begin{quote}
	"DES has four weak keys and six pairs of semi-weak keys" \cite[p.~257]{Menezes1997}.
\end{quote}
What this means is that DES can create identical sub-keys, which then also causes the encryption and decryption function to coincide. 
Furthermore if an attacker get hold of large amounts of cipher - and plain-text, an analytical attack would be possible.
\vspace{0.5em}\newline
Another issue, which may not be that relevant for this question, is also the aspect of social engineering.
To use a classic comic strip as an example - the visualization is that hackers would spend millions on dollars trying to build the "perfect" machine.
When in fact they would just go to a tool shop, buy a \$5 hammer, and beat up the person with the right access. 
Therefore one should also invest just as much into "human security" as hardware/software security.

\section{DES bit complement}
-

\begin{comment}
DES has an incredible property concerning the bitwise complement of the input and output bits. The
complement (meaning all Bits are inverted) of a bit stream A is given with A0(example: If A = 0110
the complement is A0 = 1001.) “bXOR“ is the bitwise XOR. We want to show that: If
y = DES_k(x)
than also is:
y_0 = DES_k0 (x0):
Meaning if we sent the complement of the message and the key than the complement of the original
cipher text is returned. Your exercise is to proof this property. (Hint: Try to proof the property for any
round instead of calculating all 16 rounds!)
\end{comment}
