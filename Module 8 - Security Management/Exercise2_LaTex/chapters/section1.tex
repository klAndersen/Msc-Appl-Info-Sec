\section{Two classes of breach Probability Functions}
\label{sec:Task1}
\begin{comment}
Gordon and Loeb consider two classes of breach probability functions.

1) Do the above functions capture opportunistic or strategic attackers?

2) Indicate the decision variables and parameters in these functions, and
explain their meaning with your own words.

3) Discuss what happens to SII(z; v) when v = 1. Particularly, state
what impact changes to the decision variable and parameter have on
the breach probability. Briefly argue whether or not this is realistic.

4) Recall the "expected net benefit of information security investment".
\end{comment}

\begin{equation}
S^{I}(z, v) = \frac{v}{(\alpha \cdot z + 1)^{\beta}}
\end{equation}

\begin{equation}
S^{II}(z, v) = v^{\alpha \cdot z + 1}
\end{equation}

\section{$S^{I}$(z, v) with $\alpha = 1$}
\label{sec:Task2}

\begin{equation}
S^{I}(z, v) with \alpha = 1
\end{equation}

\begin{comment}
Consider the breach probability function S^I(z; v) with alpha = 1.

1) What is the optimal security investment level?

2) You are considering to buy one of two technologies. Option 1 is modeled
with  = 2, while option 2 is modeled with  = 3. If both
technologies are available, which option do you prefer and why?

3) Visualize the optimal security investment level for both options over
v 2 [0:01; 1], given a loss of L = 100 units.

4) What percentage of the loss due to security failure should you spend
on information security when using option 1 respectively 2 for v = 1?

5) What are the expected net benefits of security for options 1 and 2
when making the optimal security investment, given v = 1?

6) Suppose that your firm is currently using a security technology that
can be modeled by option 1, and without spending any money on
security the firm will definitely suffer a loss. If you instead switch to
a technology that can be modeled by option 2, should you increase
or decrease the amount of money you spend on security? Why?

7) Consider the following setup: v = 0:5,  = 0:2, and  = 2. How
much do you have to invest into information security, such that the
probability of security breaches SI(z; v) becomes 20%?
\end{comment}